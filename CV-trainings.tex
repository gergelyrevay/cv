
%Section: Scholarships and additional info
\section{Trainings and Certificates}

\Training{SANS FOR 508: Advanced Incident Response, Threat Hunting, and Digital Forensics}
{Feb 2019}
{Incident investigation, memory analysis, filesystem analysis, lateral movement, timeline analysis. For more information visit: \newline \url{https://www.sans.org/course/advanced-incident-response-threat-hunting-training}}

\Training{SANS ICS 410: ICS/SCADA Security Essentials}
{April 2018}
{Bootcamp for ICS environments and their security. For more information visit: \newline \url{https://www.sans.org/course/ics-scada-cyber-security-essentials}}

\Training{Hardware Hacking Workshop, Tactical Network Solutions}
{Jan 2018}
{This workshop focuses on low level hardware attacks. For more information visit:\newline \url{https://www.tacnetsol.com/p/registration}}

\Training{SANS FOR 610: Reverse Engineering Malware (GREM Certified)}
{March 2017}
{This training focuses on malware reverse engineering from binary to JavaScript. For more information visit: \newline \url{https://www.sans.org/course/ reverse-engineering-malware-malware-analysis-tools-techniques}}

\Training{The ARM Exploit Laboratory by Saumil Shah, 0xA RECON}
{June 2016}
{The class covers everything from an introduction to ARM assembly all the way to Return Oriented Programming (ROP) on ARM architectures. For more information visit:\newline \url{https://recon.cx/2016/ training/trainingexploitlab.html.}}

\Training{SANS SEC 575: Mobile Device Security and Ethical Hacking (GMOB Certified)}
{April 2015}
{Hands-on course on the security of mobile devices, particularly on IOS and Android. For more information visit:\newline \url{http://www.sans.org/course/mobile-device-security-ethical-hacking}}

\Training{SANS SEC 760: Advanced Exploit Development for Penetration Testers }
{July 2014}
{It teaches the skills required to reverse-engineer 32-bit and 64-bit applications, perform remote user application and kernel debugging, analyze patches for 1-day exploits, and write complex exploit, such as use-after-free attacks against modern software and operating systems.For more information visit:\newline \url{http://www.sans.org/course/advance-exploit-development-pentetration-testers}}

\Training{Offensive Security Certified Professional (OSCP Certified)}
{May 2013}
{In-depth technical training course which covers the techniques and the whole process of penetration testing from the initial information gathering through exploitation and post exploitation techniques to writing a penetration testing report. For more information download the syllabus from \url{http://www.offensive-security.com/documentation/penetration-testing-with-backtrack.pdf}}


\Training{One week training on writing scientific papers at \textsc{Telecom ParisTech}, Paris, France}
{May 2009}
{\url{http://www.telecom-paristech.fr} \newline The training went through the research work in general, how to choose a topic, how a research process should go and how to write different scientific papers.}

\Training{One week training on international team management}
{Nov 2008}
{\url{http://www.telecom-paristech.fr} \newline The training dealt with the communication and management of multinational teams.}

\Training{Computer Administrator qualification (OKJ 52464103) at \textsc{Iron- and Electrical Industrial Technical High School}, Sopron, Hungary}
{June 2003}
{As the end of the IT studies in high school we took this exam which includes network and desktop administration and deep understanding of the Office programs.}
